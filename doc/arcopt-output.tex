\documentclass[11pt]{article}

\input{/home/nwh/Dropbox/templates/nwh-style.sty}

\title{Arcopt output summary}
\author{Nick Henderson}
%\date{}

\newcommand{\code}[1]{\texttt{#1}}

\begin{document}
\maketitle

\section{Example}

\begin{verbatim}
 iter fevcnt        f   ||z|| D     eval M s scnt    sres     stp bas bfac
    1      2 -1.2e-01 1.8e-02   -1.3e+00 m 0   11 2.7e-10 2.7e-01  nb u
    2      3 -1.4e-01 3.3e-01   -4.7e+00   0    9 6.6e-08 4.3e-02  nb u
    3      4 -1.8e-01 2.7e-01   -2.3e+00   0    9 9.4e-12 9.4e-02  nb u
    4      5 -2.4e-01 3.0e-01   -2.7e+00   0    8 9.1e-12 1.2e-01  nb u
    5      6 -2.5e-01 3.6e-01   -3.5e+00   0    7 1.4e-11 9.4e-03  nb u
    6      7 -3.1e-01 3.0e-01   -4.1e+00   0    6 1.3e-11 5.4e-02  nb u
    7      8 -4.2e-01 6.7e-01   -2.0e+00   0    5 1.7e-11 1.2e-01  nb u
    8      9 -4.5e-01 7.1e-01   -2.4e+00   0    4 2.2e-11 2.0e-02  nb u
    9     10 -5.9e-01 6.9e-01   -1.8e+00   0    3 4.4e-12 2.3e-01  nb u
   10     11 -9.4e-01 1.0e+00   -1.6e+00 m 0    2 1.3e-12 4.7e-01  nb u
\end{verbatim}

\section{Column definitions}

\begin{description}
\item[\code{iter}] Iteration
\item[\code{fevcnt}] Function evaluation counter
\item[\code{||z||}] Reduced gradient
\item[\code{D}] Primal degeneracy flag.  A ``\code{d}'' appears in this column
  if the current iterate is found to be degenerate.
\item[\code{eval}] Minimum eigenvalue of reduced Hessian as reported by
  Matlab's \code{eigs}.  The eigenvector corresponding to this eigenvalue is
  used as a direction of negative curvature.
\item[\code{M}] Modification or perturbation flag.  If ``\code{m}'' appears in
  this column, then $|g_k\T H_k d_k| \le \tau |d_k\T H_k d_k|$ with $0 < \tau <
  \frac{1}{3}$.  In this case the ``right hand side'' vector for the arc is set
  to $g_k + d_k$.  This provides perturbation and allows the proof of
  convergence to points satisfying the second order necessary optimality
  conditions.  In the linearly constrained case the vectors refer to their reduced
  counterparts.
\item[\code{s}] Flag returned from iterative linear solver.  The code $0$
  indicates success.
\item[\code{scnt}] Number of matrix vector products used by linear solver
\item[\code{sres}] Final residual from iterative linear solver
\item[\code{stp}] Step size selected by arc search
\item[\code{bas}] Basis change flags.  ``\code{s}'' indicates that a variable
  has been made super-basic (a constraint has been deleted).  ``\code{n}''
  indicates that a variable has become non-basic (a constraint has been
  added).  ``\code{b}'' indicates a change to the basic set of variables.  Note
  that there is an option in the code to prevent constraint deletions right
  after additions.  This setting is off for these problems, because cycling
  does not appear to be an issue.
\item[\code{bfac}] Basis factorization flags.  ``\code{u}'' indicates rank-1
  update.  ``\code{f}'' indicates refactorization of the basis.  ``\code{s}''
  indicates BS repair.  ``\code{r}'' indicates BR repair.
\end{description}

% fix reference to bibtex file
\bibliography{/home/nwh/Dropbox/ref/ref}{}
\bibliographystyle{plain}

\end{document}